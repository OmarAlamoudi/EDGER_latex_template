
\documentclass{edger}


\begin{document}
\title{An example EDGER paper in latex
}

\shorttitle{Example Manuscript}

\authors{Author1, Author2, and Author3}

\institution{}

%\begin{mdframed}[style=mdfexample1]
\begin{edgerabstract}
The \verb|\edgerabstract| environment formats and aligns the abstract and draws a border around it. Do not worry if your abstract is longer than one page, the environment would extend the box to multiple pages. 

Blind text follows: \lipsum[1]
\end{edgerabstract}



\section*{Introduction}

\lipsum[1-7]   


\section*{Figures}

Single figures can be plotted through the \verb|\plot| command. The syntax goes as

 \verb|\plot{strue}{width=0.5\textwidth}{A sample figure caption for a sample figure }| .
 The figure can be referred as Figure~\ref{fig:strue} using \verb|\ref{fig:strue}|

\plot{strue}{width=0.5\textwidth}{A sample figure caption for a sample figure }.


Blind Text Follows
\lipsum[1-7]  
\subsection*{Multiple  figures}

Multiple figures are implemented with the \verb|\subfigure| environment. 
\begin{figure}
	\centering
	\begin{subfigure}[c]{0.495\textwidth}
\multiplot{strue}{width=0.99\textwidth,height=0.5\textwidth}{}
	\end{subfigure}
	\begin{subfigure}[c]{0.495\textwidth}
\multiplot{sstart}{width=0.99\textwidth,height=0.5\textwidth}{}
	\end{subfigure}
	\begin{subfigure}[c]{0.495\textwidth}
\multiplot{sinv}{width=0.99\textwidth,height=0.5\textwidth}{}
	\end{subfigure}
	\begin{subfigure}[c]{0.495\textwidth}
\multiplot{sinv_copy}{width=0.99\textwidth,height=0.5\textwidth}{}
	\end{subfigure}
\caption{(a) One figure (b) Second figure (c) Third figure and (d) Fourth figure}	
\label{fig:bigfig}
\end{figure}


Individual figures can be accessed as Figure~\ref{fig:sstart} and single figure can be referred as Figure~\ref{fig:bigfig}.

\lipsum[1-7]  
\section*{Table}
A table is implemented using the \verb|\tabl| command with an example shown below adapted from segtex class.
\tabl{example}{This table is specified in the document by \texttt{
		$\backslash$tabl\{example\}\{This caption.\}\{\ldots\}}.
}{
	\begin{center}
		\begin{tabular}{|c|c|c|}
			\hline
			\multicolumn{3}{|c|}{Table Example} \\
			\hline
			migration\rule[-2ex]{0ex}{5ex} & 
			$\omega \rightarrow k_z$ & 
			$k_y^2+k-z^2\cos^2 \psi=4\omega^2/v^2$ \\
			\hline
			\parbox{1in}{zero-offset\\diffraction}\rule[-4ex]{0ex}{8ex} &
			$k_z\rightarrow\omega_0$ &
			$k_y^2+k_z^2=4\omega_0^2/v^2$ \\
			\hline
			DMO+NMO\rule[-2ex]{0in}{5ex} & $\omega\rightarrow\omega_0$ & 
			$\frac{1}{4}
			v^2k_y^2\sin^2\psi+\omega_0^2\cos^2\psi=\omega^2$ \\
			\hline
			radial DMO\rule[-2ex]{0in}{5ex} & $\omega\rightarrow\omega_s$ &
			$\frac{1}{4}v^2k_y^2\sin^2\psi+\omega_s^2=\omega^2$\\
			\hline
			radial NMO\rule[-2ex]{0in}{5ex} & $\omega_s\rightarrow\omega_0$ &
			$\omega_0\cos\psi=\omega_s$\\
			\hline
		\end{tabular}
	\end{center}
}

The table can be referred to as Table~\ref{tabl:example}.

\lipsum[1-7]  
\section*{equations}

Equations are implemented using the \verb|\equation| environment. An example is given below

\begin{equation}
\begin{bmatrix}
\mathbf{M}_{bb} & \mathbf{M}_{ba} \\
\mathbf{M}_{ab} & \mathbf{M}_{aa}
\end{bmatrix}
\begin{bmatrix}
\mathbf{p}_{b}  \\
\mathbf{p}_{a} 
\end{bmatrix}
=
\begin{bmatrix}
\mathbf{s}_{b}  \\
\mathbf{s}_{a} 
\end{bmatrix}
\label{eqn:system}
\end{equation}

The equation can be referred to as equation~\ref{eqn:system}.

\lipsum[1-7]  

\section*{Referring intext}
In-line references are written as \cite{knuth1989} while in-text references can be referred with \cite[]{knuth1989}. The references are based on the seg style after the segtex format.

Re-compiling after adding new references can lead to misssing references. I have added a script \verb|refscript.sh| which should be run after adding new references that rebuilds the reference index and new references are added. 

\bibliographystyle{seg}  % style file is seg.bst
\bibliography{edger}

\end{document} 